% Paper template for TAR 2016
% (C) 2014 Jan Šnajder, Goran Glavaš, Domagoj Alagić, Mladen Karan
% TakeLab, FER

\documentclass[10pt, a4paper]{article}

\usepackage{tar2016}

\usepackage[utf8]{inputenc}
\usepackage[pdftex]{graphicx}
\usepackage{booktabs}
\usepackage{amsmath}
\usepackage{amssymb}

\title{Shit-of-the-art humor detection}

\name{Bartol Freškura, Filip Gulan, Damir Kopljar}

\address{
University of Zagreb, Faculty of Electrical Engineering and Computing\\
Unska 3, 10000 Zagreb, Croatia\\
\texttt{\{bartol.freskura, filip.gulan, damir.kopljar\}@fer.hr}\\
}

\abstract{ 
This is the paper abstract.
}

\begin{document}

\maketitleabstract

\section{Introduction}
Introduction

\section{Related Work}

In scientific papers, this section usually (but not necessarily) briefly describes the related research. 

\section{Our model architecture}

The paper should have a minimum of 3 and a maximum of 5 pages, plus an additional page for references.

\subsection{Recurrent Neural Networks}

\subsection{Convolutional Neural Networks}


\section{Data}

\subsection{Figures}

Here is an example on how to include figures in the paper. Figures are included in \LaTeX{} code immediately \textit{after} the text in which these figures are referenced. Allow \LaTeX{} to place the figure where it believes is best (usually on top of the page of at the position where you would not place the figure). Figures are referenced as follows: ``Figure~\ref{fig:figure1} shows \dots''. Use tilde (\verb.~.) to prevent separation between the word ``Figure'' and its enumeration. 

\begin{figure}
\begin{center}
\includegraphics[width=\columnwidth]{drawing.pdf}
\caption{This is the figure caption. Full sentences should be followed with a dot. The caption should be placed \textit{below} the figure. Caption should be short; details should be explained in the text.}
\label{fig:figure1}
\end{center}
\end{figure}

\subsection{Tables}

There are two types of tables: narrow tables that fit into one column and a wide table that spreads over both columns.

\subsubsection{Narrow tables}

Table~\ref{tab:narrow-table} is an example of a narrow table. Do not use vertical lines in tables -- vertical tables have no effect and they make tables visually less attractive.

\begin{table}
\caption{This is the caption of the table. Table captions should be placed \textit{above} the table.}
\label{tab:narrow-table}
\begin{center}
\begin{tabular}{ll}
\toprule
Heading1 & Heading2 \\
\midrule
One & First row text \\
Two   & Second row text \\
Three   & Third row text \\
      & Fourth row text \\
\bottomrule
\end{tabular}
\end{center}
\end{table}

\subsection{Wide tables}

Table~\ref{tab:wide-table} is an example of a wide table that spreads across both columns. The same can be done for wide figures that should spread across the whole width of the page. 

\begin{table*}
\caption{Wide-table caption}
\label{tab:wide-table}
\begin{center}
\begin{tabular}{llr}
\toprule
Heading1 & Heading2 & Heading3\\
\midrule
A & A very long text, longer that the width of a single column & $128$\\
B & A very long text, longer that the width of a single column & $3123$\\
C & A very long text, longer that the width of a single column & $-32$\\
\bottomrule
\end{tabular}
\end{center}
\end{table*}

\section{Experiments}

Math expressions and formulas that appear within the sentence should be written inside the so-called \emph{inline} math environment: $2+3$, $\sqrt{16}$, $h(x)=\mathbf{1}(\theta_1 x_1 + \theta_0>0)$. Larger expressions and formulas (e.g., equations) should be written in the so-called \emph{displayed} math environment:

\[
b^{(i)}_k = \begin{cases}
1 & \text{if 
    $k = \text{argmin}_j \| \mathbf{x}^{(i)} - \mathbf{\mu}_j \|$}\\
0 & \text{otherwise}
\end{cases}
\]

Math expressions which you reference in the text should be written inside the \textit{equation} environment:

\begin{equation}\label{eq:kmeans-error}
J = \sum_{i=1}^N \sum_{k=1}^K 
b^{(i)}_k \| \mathbf{x}^{(i)} - \mathbf{\mu}_k \|^2
\end{equation}

Now you can reference equation \eqref{eq:kmeans-error}. If the paragraph continues right after the formula

\begin{equation}
f(x) = x^2 + \varepsilon
\end{equation}

\noindent like this one does, use the command \emph{noindent} after the equation to remove the indentation of the row. 

Multi-letter words in the math environment should be written inside the command \emph{mathit}, otherwise \LaTeX{} will insert spacing between the letters to denote the multiplication of values denoted by symbols. For example, compare
$\mathit{Consistent}(h,\mathcal{D})$ and\\
$Consistent(h,\mathcal{D})$.

If you need a math symbol, but you don't know the corresponding \LaTeX{} command that generates it, try
\emph{Detexify}.\footnote{\texttt{http://detexify.kirelabs.org/}}

\section{Conclusion}

Conclusion is the last enumerated section of the paper. It should not exceed half of a column and is typically split into 2--3 paragraphs. No new information should be presented in the conclusion; this section only summarizes and concludes the paper.

\section*{Acknowledgements}

Hvala svima u studiju i režiji.

\bibliographystyle{tar2016}
\bibliography{tar2016} 

\end{document}

